%%%%%%%%%%
% to je komentar
% vsi ukazi imajo sledečo obliko:
% \ukaz{option}

%%%%%%%%%%
% pri article classu se stevilka strani avtomatsko doda
% sledeca vrstica je ukaz
\documentclass{article}

%%%%%%%%%%
% sledi "obmocje" PREAMBLE ki se ne izpise direktno v dokumentu.
% je par ukazov, ki se kasneje uporabijo za avtomatsko ustvarjenje naslovnice
    \title{PRVI NA SCENI}
    \date{04.04.2019}
    \author{rezlou soimon}

%%%%%%%%%%
% za dodajanje "packages" dodaj ukaz v preamble datoteke
% \usepackage{packagename}
\usepackage{amsmath}


%%%%%%%%%%
% sledi enviroment ukaz. mora met tud \end{} na koncu. znotraj nekega enviromenta veljajo dolocena, nastavljena pravila
\begin{document}

%%%%%%%%%%
% zelo uporabno tkole dodat naslovnico
    \pagenumbering{gobble}    % no numbering
        \maketitle
        \newpage
    \pagenumbering{roman}     % rimske cifre 

%%%%%%%%%%
% nove vrstice v editorju se ne izpisejo kot nove vrstice v dokumentu
asd
assdasdasdasda
da as dasdas 

%%%%%%%%%%
        \newpage    % gre na naslednjo, novo stran
    \pagenumbering{arabic}     % nase navadne cifre

%%%%%%%%%%
\section{SECTION1}
Hello World!

\subsection{subsect1}
to je moj prvi dokument v Latexu šššššččččččč

% paragrafi se ne dodajo v kazalo, sectioni pa se
 \paragraph{nekiiiii}

\subparagraph{ullalaaaa}

%%%%%%%%%
% vse kar je znotraj okolja za enacbe se prevede kot enacba in tudi ostevilci
% ker sem dol snel "amsmath" lahko zdej neostevilcene enacbe nardim.
\begin{equation*}    % neostevilceno
  f(x) = x^2+3^5
\end{equation*}

\begin{equation}    %ostevilcena enacba
  f(x) = (x^2+3^5)/x^2
\end{equation}


\end{document}













